\documentclass[11pt]{article}
\usepackage{geometry}
\usepackage{amsmath}
\usepackage{hyperref}
\usepackage{amssymb}

\begin{document}

\begin{center}
  {\LARGE \textbf{Project Proposal: Leader–Follower Particle System with Capacity Constraints}}\\[1em]
  {\large Zitian Wang}
\end{center}

\section*{1. Objectives}
\begin{enumerate}
  \item Implement from scratch the classic Vicsek alignment model as a baseline.
  \item Extend it by introducing:
    \begin{itemize}
      \item Randomly entering leader particles in a 2D domain.
      \item A capacity limit $n = N_{\text{student}} / N_{\text{leader}}$ on how many followers each leader can influence.
      \item Nearest-leader assignment when multiple leaders compete for the same student.
      \item Periodic inter-leader communication to adjust headings for a more uniform spatial spread.
    \end{itemize}
  \item Define and measure convergence through:
    \begin{itemize}
      \item Polarization of followers $\Phi$.
      \item Stable leader–follower assignments.
      \item Balanced follower loads.
      \item Low variance in inter-leader distances.
    \end{itemize}
  \item Compare the extended model against the pure Vicsek model in terms of convergence speed, final formations, and robustness.
\end{enumerate}

\section*{2. Methodology}
\subsection*{2.1 Baseline Vicsek Model}
\begin{itemize}
  \item Particles $i=1,\dots,N$ each have position $\mathbf{x}_i(t)\in\mathbb{R}^2$ and heading $\theta_i(t)$.
  \item Heading update:
    \[
      \theta_i(t+1)
      = \mathrm{Arg}\!\Bigl\{\sum_{j\in\mathcal{N}_i} e^{i\theta_j(t)}\Bigr\} + \Delta_i(t),
      \quad \Delta_i\sim\mathrm{Unif}\bigl[-\tfrac{\eta}{2},\tfrac{\eta}{2}\bigr].
    \]
  \item Position update:
    \[
      \mathbf{x}_i(t+1)
      = \mathbf{x}_i(t) + v_0\bigl(\cos\theta_i(t+1),\,\sin\theta_i(t+1)\bigr).
    \]
\end{itemize}

\subsection*{2.2 Leader–Follower Extension}
\begin{itemize}
  \item \textbf{Leader spawning:} $L$ leaders appear at random times/locations.
  \item \textbf{Capacity constraint:} Each leader may attract at most $n$ students.
  \item \textbf{Assignment rule:} Students choose the closest leader with available capacity.
  \item \textbf{Follower update:} At each step, a follower aligns with the average heading of its assigned leader and Vicsek neighbors.
\end{itemize}

\subsection*{2.3 Inter-Leader Communication}
\begin{itemize}
  \item Every $T$ time steps, leaders exchange positions.
  \item Each leader adjusts its heading to reduce local leader density (e.g.\ repulsion from nearest leaders).
  \item Between communications, leaders move according to their updated headings.
\end{itemize}

\section*{3. Convergence \& Evaluation Metrics}
\begin{itemize}
  \item \textbf{Polarization:} 
    \[
      \Phi(t) = \frac{1}{N}\Bigl\lvert\sum_{i=1}^N\mathbf{v}_i(t)\Bigr\rvert,\quad
      \mathbf{v}_i = (\cos\theta_i,\sin\theta_i).
    \]
    Convergence when $\Phi\ge0.95$ over several steps.
  \item \textbf{Assignment stability:} Fraction of students whose leader assignment remains unchanged over 10 steps.
  \item \textbf{Load balance:} Standard deviation of follower counts across leaders below a small threshold.
  \item \textbf{Leader spacing:} Variance of pairwise leader distances falls below a threshold.
\end{itemize}

\section*{4. Expected Outcomes}
\begin{itemize}
  \item A clean, original Python implementation of both the baseline and extended models.
  \item Visualizations (animations and plots) showing:
    \begin{itemize}
      \item Pure Vicsek alignment.
      \item Leader-driven clustering.
      \item Effects of capacity constraints.
      \item Inter-leader dispersal via communication.
    \end{itemize}
  \item A comprehensive report detailing methods, results, and potential future extensions (e.g.\ dynamic capacities, obstacle avoidance).
\end{itemize}

\bigskip
\noindent Please let me know if this proposal aligns with your expectations or if you would suggest any modifications.

\bigskip
\noindent Best regards,\\
Zitian Wang

\end{document}
