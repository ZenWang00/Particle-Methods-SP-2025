\documentclass[11pt]{article}
\usepackage[utf8]{inputenc}
\usepackage{amsmath,amssymb}
\usepackage{graphicx}
\usepackage{hyperref}
\usepackage{siunitx}
\title{Implementation Plan for "AddBiomechanics Dataset: Capturing the Physics of Human Motion at Scale"}
\author{Zitian Wang}
\date{\today}
\begin{document}
\maketitle

\section*{1. Background and Methods}
The paper introduces the AddBiomechanics dataset, a large-scale repository of synchronized motion-capture, force-plate, and medical imaging data from hundreds of human subjects performing various activities of daily living (walking, running, jumping).  The dataset includes:
\begin{itemize}
  \item 3D marker trajectories (200+ markers, 120~Hz)
  \item Ground reaction forces (3D, 1000~Hz)
  \item Synchronized MRI scans of musculoskeletal anatomy
\end{itemize}
The authors demonstrate how inverse dynamics driven by motion data can estimate joint torques and muscle activations, and they validate these estimates against EMG measurements.

\section*{2. Implementation Plan}
\subsection*{2.1 Reproducible Results}
We will reproduce the published walking trial on a treadmill at 1.25~m/s: joint angle trajectories and computed knee joint torque profiles.  Targets: match peak knee extension torque (\SI{1.2}{Nm/kg}) within 10\% and reproduce waveform shape (correlation $>0.95$).

\subsection*{2.2 Data \& Parameter Settings}
\begin{itemize}
  \item Subjects: select 10 healthy adults (balanced gender, age 20--40).
  \item Trials: treadmill walking at 1.25~m/s, 60~s duration.
  \item Marker data: low-pass filter at 6~Hz (Butterworth, 4th order).
  \item Force data: low-pass filter at 20~Hz.
  \item Inverse dynamics:
    \begin{itemize}
      \item Segment inertial parameters from anthropometric tables (Dempster).
      \item Joint centers estimated from marker clusters.
    \end{itemize}
  \item EMG comparison: normalize EMG amplitude to %MVC.
\end{itemize}

\subsection*{2.3 Model Extensions and New Problems}
\begin{itemize}
  \item \textbf{Particle-based muscle modeling}: integrate SPH muscle fibers into kinematic chain to predict stress distributions.
  \item \textbf{Real-time feedback}: implement a particle-fluid hybrid to simulate blood flow and heat dissipation under dynamic loading.
  \item \textbf{Pathological analysis}: compare joint kinetics for subjects with gait impairments, to identify biomarkers of early osteoarthritis.
  \item \textbf{Data augmentation}: use generative particle models to synthesize new motion patterns for machine-learning applications.
\end{itemize}

\vfill
\noindent\textit{This plan leverages the multimodal AddBiomechanics dataset to validate and extend particle-based musculoskeletal simulations at scale.}
\end{document} 